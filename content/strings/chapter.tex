\chapter{Strings}


\section{Applications of String Functions}

{\scriptsize
\subsection*{\small{Searching for a Sub-string in a String (KMP)}}
\begin{itemize}
    \item Concatenate the sub-string to the beginning of the string plus another character that does not occur in either string.
    \item Run the prefix function on the string and count the number of occurrences of the length of the original sub-string.
\end{itemize}

\subsection*{\small{Counting the Number of Prefixes in a String}}
\begin{itemize}
    \item Run the prefix function on the string.
    \item For each string of length $i$, do:
        \text{ans[pi[i-1]] += ans[i]}
    since each occurrence of a longer prefix implies the occurrence of the shorter prefix.
    \item Add one for each original (non-counted) prefix.

    vll ans(n+1, 1); // Set to one for the actual prefix
    for (int i = 0; i < n; i++) ans[pi[i]]++;
    for (int i = n-1; i > 0; i--) ans[pi[i-1]] += ans[i];
\end{itemize}


\subsection*{\small{The Number of Different Substrings in a String}}
\begin{itemize}
    \item Start with an empty string.
    \item Add the next character of the original string to our string and reverse the string.
    \item Run the prefix function to find the max length of the substrings we have already found.
    \item The number of new substrings = accumulate$(|S| + 1 - \text{maxPi})$.
\end{itemize}

\subsection*{\small{Compressing a String}}
\begin{itemize}
    \item Given a string $s$ of length $n$, we want to find the shortest ``compressed'' representation of the string.
    \item If after running the prefix function, $n$ is divisible by $(n - \text{pi[n-1]})$, then the string is compressible, with minimum size $(n - \text{pi[n-1]})$.
    \item Otherwise, it is not compressible.
\end{itemize}

\subsection*{\small{Creating Automation for String Matching}}
\begin{itemize}
    \item After running the prefix function on one or many strings, an automation should be able to preprocess the strings in $O(n)$ and then do matching in $O(m)$ per query.
    \item This is especially useful for \textbf{many repeated searches} with the same pattern (KMP runs in $O(n+m)$ per query).
    \item Build a table storing \texttt{aut[state][charNumber]}, where \texttt{state} is the index of the string we are searching for and \texttt{char} is the current character.
    \item Depending on the character we see, move forwards or backwards to a new state (e.g., seeing an \texttt{a} moves us from state 1 → 2, but seeing a \texttt{b} moves us from state 2 → 0).
\end{itemize}
}


% Import KACTL snippets
\kactlimport{MatchingAutomation.h}
\kactlimport{KMP.h}
\kactlimport{Zfunc.h}
\kactlimport{Manacher.h}
\kactlimport{Eertree.h}
\kactlimport{MinRotation.h}
\kactlimport{SuffixArray.h}
\kactlimport{SuffixAutomaton.h}
\kactlimport{SuffixTree.h}
\kactlimport{Hashing.h}
\kactlimport{HashInterval.h}
\kactlimport{AhoCorasick.h}
\kactlimport{LyndonFactorization.h}
\kactlimport{SuffixAutomaton.h}
